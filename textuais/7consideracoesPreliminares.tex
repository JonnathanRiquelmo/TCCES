%#################################################################
\chapter{Considerações Finais}\label{consideracoesFinais}
%#################################################################

Este \ac{TCC} se propôs a realizar um estudo que abordou a área de projeto e modelagem conceitual de \acp{BD} relacionais.
Para tanto, foi necessário realizar um desenho de pesquisa que orientou toda a execução das atividades necessárias para atingir os objetivos específicos.
O projeto envolveu a pesquisa bibliográfica para o embasamento teórico, a condução de um \ac{MLM} visando encontrar estudos relacionados, o desenvolvimento da solução proposta e um experimento controlado para sua avaliação.

Através dos estudos realizados, assim como por meio dos resultados obtidos com o \ac{MLM} e do experimento controlado, é possível concluir que o objetivo principal deste trabalho foi alcançado.
Isto significa que se ofereceu uma alternativa ferramental para uso de uma abordagem textual para apoiar o processo de ensino-aprendizagem de projeto de modelagem de \ac{BD} relacionais. Além disso, mediante uma possível exposição contínua da gramática aos alunos, apresentou-se uma perspectiva de melhora em relação ao esforço associado. 

Além disso, destaca-se como principais contribuições deste \ac{TCC}:

\begin{itemize}
    \item A investigação da literatura científica e o relato de experiência, adquirida principalmente durante a etapa de projeto, acerca da forma de desenvolvimento de uma \ac{DSL} textual utilizando um \ac{LW};
    \item A definição da gramática de uma \ac{DSL} voltada à modelagem conceitual de \acp{BD} relacionais;
    \item A avaliação preliminar do protótipo e sua evolução;
    \item O uso de tecnologias \textit{open source};
    \item A Avaliação empírica do produto desenvolvido;
        \begin{itemize}
            \item Definição de um protocolo experimental;
            \item Execução;
            \item Coleta de dados;
            \item Análise quantitativa e qualitativa.
        \end{itemize}
    \item A disponibilização do \textit{plugin} da linguagem desenvolvida, com suporte à modelagem conceitual, bem como mapeamento e transformação para o modelo lógico.
\end{itemize}

Inegavelmente, houveram diversos desafios durante o andamento deste \ac{TCC}. 
A definição do protocolo de \ac{MLM} e do experimento não foram atividades triviais, pois exigem experiência para que se consiga obter resultados confiáveis.
O desenvolvimento da solução também se mostrou dificultosa em momentos, principalmente na implementação do gerador que faz o mapeamento do modelo conceitual para o modelo lógico. 
Entretanto, por meio do uso de metodologias apoiadas na literatura, e também de protocolos bem definidos, foi possível superar as adversidades enfrentadas.

Outro ponto que merece destaque foi o esforço dos pesquisadores envolvidos neste estudo para a realização de três artigos.
O primeiro abordou o \ac{SLM} e foi submetido ao \textit{ER - International Conference on Conceptual Modeling} (ER'2019).
Outro abrangeu toda o \ac{MLM} e foi submetido ao Simpósio Brasileiro de Bancos de Dados (SBBD'2019). 
E finalmente, um artigo descrevendo o protótipo desenvolvido, o qual foi submetido à Escola Regional de Engenharia de Software (ERES'2019), aprovado e apresentado.

Existe a intenção de que seja realizada a escrita de outro artigo, desta vez compilando os resultados de todo o estudo, a ser submetido à um \textit{journal} inserido na área deste trabalho.

% Como principais resultados obtidos está a investigação conduzida para compreender o estado da arte e da prática no uso de \acp{DSL} para modelagem de \acp{BD}, bem como a criação de uma \ac{DSL} textual utilizando o \textit{framework} Xtext, ferramenta \textit{open source} que permitiu a integração do protótipo a um \ac{RCP} Eclipse. 

% Outro ponto que merece destaque foi o esforço dos pesquisadores envolvidos neste estudo para a realização de dois artigos, sendo um abordando o \ac{SLM} e submetido ao \textit{ER - International Conference on Conceptual Modeling} (ER'2019), e outro abrangendo toda o \ac{MLM}, o qual foi submetido ao Simpósio Brasileiro de Bancos de Dados (SBBD'2019). Ambos os trabalhos estão em fase de avaliação.

% Pretende-se que ao final do trabalho esta proposta não apenas realize modelagem, mas também a transformação dos modelos \ac{ER} gerados em \textit{scripts} \ac{SQL} para diferentes tecnologias \acp{SGBD}, \textit{e.g.} PostgreSQL, MySQL e SQL Server.

%#################################################################
\section{Trabalhos Futuros}
%#################################################################


Como sugestão de trabalhos futuros é possível indicar que através dos resultados do experimento executado foi identificada a necessidade de evolução da \ac{DSL} desenvolvida. 
A evolução pode ser tanto em cima dos construtores atualmente suportados, quanto na definição de novos termos para cobrir outros conceitos importantes do modelo \ac{ER}.

Também pode-se citar que existe espaço para a aplicação das regras de transformação entre modelos ainda não desenvolvidas, bem como a geração de \textit{scripts} \ac{SQL} que para tecnologias específicas \textit{e.g.} PostgreSQL, MySQL e SQL Server.

% Na fase atual desta pesquisa pode-se dizer que foram obtidos avanços, porém ainda é necessário realizar a finalização de diversos aspectos da proposta. 
% Os principais pontos que devem ser investigados e desenvolvidos dizem respeito à transformação completa dos modelos conceituais para modelos lógicos, bem como a geração de \textit{scripts} \ac{SQL} que representem o modelo físico.

% Pretende-se resolver estas questões com o andamento do trabalho e, com isto feito, realizar uma avaliação experimental da proposta. 
% É importante salientar que a \ac{DSL} será avaliada preliminarmente ainda no começo do \ac{TCC} II, visando assim obter o refinamento da gramática da linguagem criada.

% %#################################################################
% \section{Cronograma}
% %#################################################################

% Para o restante da execução da pesquisa, o desenvolvimento deste trabalho se dará conforme as atividades descritas na \autoref{tbl:cronograma}.


% \begin{landscape}
% \definecolor{midgray}{gray}{.5}
% \begin{table}[!htb]
%     \caption{Cronograma do Trabalho de Conclusão de Curso.}
%     \label{tbl:cronograma}
% 	\centering
% 		\begin{tabular}{l|c|c|c|c|c|c|c|c|c}
% 		\bottomrule
% 		\rowcolor[HTML]{C0C0C0}
% 		\textbf{ATIVIDADE}&\multicolumn{5}{c|}{\large\textbf{2019/1}}&\multicolumn{4}{c}{\large\textbf{2019/2}}\\
% 		\hline
% 		\rowcolor[HTML]{C0C0C0}
% 		&\textbf{MAR}&\textbf{ABR}&\textbf{MAI}&\textbf{JUN}&\textbf{JUL}&\textbf{AGO}&\textbf{SET}&\textbf{OUT}&\textbf{NOV}\\
% 		\hline
% 		Elaboração da proposta de TCC &\cellcolor{Blue}&&&&&&&&\\
% 		\hline
% 		Pesquisa bibliográfica da fundamentação teórica &\cellcolor{Blue}&\cellcolor{Blue}&\cellcolor{Blue}&&&&&&\\
% 		\hline
% 		Planejamento e execução do MLM &\cellcolor{Blue}&\cellcolor{Blue}&\cellcolor{Blue}&&&&&&\\
% 		\hline	
% 		Análise de qualidade dos Resultados do MLM &&\cellcolor{Blue}&\cellcolor{Blue}&&&&&&\\
% 		\hline			
% 		Extração e análise dos dados do MLM &&\cellcolor{Blue}&\cellcolor{Blue}&&&&&&\\
% 		\hline	
% 		Implementação do protótipo &&&\cellcolor{Blue}&\cellcolor{Blue}&&&&&\\
% 		\hline
% 		Demonstração do protótipo &&&&\cellcolor{Blue}&&&&&\\
% 		\hline
% 		Escrita do TCC I &&\cellcolor{Blue}&\cellcolor{Blue}&\cellcolor{Blue}&&&&&\\
% 		\hline
% 		Aplicação das melhorias sugeridas pela banca de TCC1&&&&\cellcolor{midgray}&\cellcolor{midgray}&&&&\\
% 		\hline
% 		Planejamento da avaliação preliminar &&&&\cellcolor{midgray}&\cellcolor{midgray}&&&&\\
% 		\hline
% 		Execução da avaliação preliminar &&&&&\cellcolor{midgray}&&&&\\
% 		\hline	
% 		Análise da avaliação preliminar &&&&&\cellcolor{midgray}&\cellcolor{midgray}&&&\\
% 		\hline	
% 		Evolução do desenvolvimento &&&&&\cellcolor{midgray}&\cellcolor{midgray}&\cellcolor{midgray}&\cellcolor{midgray}&\\
% 		\hline	
% 		Planejamento da avaliação experimental &&&&&&\cellcolor{midgray}&\cellcolor{midgray}&&\\
% 		\hline	
% 		Execução da avaliação experimental &&&&&&&\cellcolor{midgray}&&\\
% 		\hline	
% 		Análise dos resultados da avaliação experimental &&&&&&&\cellcolor{midgray}&\cellcolor{midgray}&\\
% 		\hline	
% 		Escrita de artigos para submissão em eventos científicos
% 		&&&&&\cellcolor{midgray}&\cellcolor{midgray}&\cellcolor{midgray}&\cellcolor{midgray}&\cellcolor{midgray}\\
% 		\hline	
% 		Escrita do TCC II 
% 		&&&&&&\cellcolor{midgray}&\cellcolor{midgray}&\cellcolor{midgray}&\cellcolor{midgray}\\
% 		\toprule	
% 		\end{tabular}
% 		\fonte{O autor.}
% \end{table}
% \end{landscape}


