%#################################################################
\chapter{Introdução}\label{introducao}
%#################################################################

É praticamente impossível administrar o mundo moderno sem software. 
As infraestruturas de empresas nacionais e multinacionais são controladas por sistemas baseados em computadores e a maioria dos produtos elétricos inclui algum computador ou software de controle. 
A fabricação e distribuição industrial é totalmente informatizada, assim como o sistema financeiro. 
Entretenimento, incluindo a indústria criativa formada por jogos, música e cinema tem intensiva participação de software em suas atividades. 
Portanto, a \ac{ES} é essencial para o funcionamento das sociedades nacionais e internacionais \cite{Sommerville:2011}.

Contudo, a \ac{ES} é inviável sem a persistência e manipulação de dados. 
Nos primórdios do uso de computadores, a persistência de dados se dava em forma de arquivos de texto simples, inspirado inicialmente nas raízes de uma invenção muito antiga, denominada máquina de tabulação, criada por volta de 1860. 
Todavia em pouco tempo, e em razão da evolução das tecnologias, o uso de arquivos de texto começaram a se mostrar ineficientes \cite{Silberschatz:1999}. 

Uma alternativa ao problema relacionado aos arquivos de texto, surgida há mais de 40 anos atrás e ainda utilizada amplamente nos dias atuais, foi justamente a automatização do conceito das máquinas de tabulação. 
Este feito foi realizado pelo cientista da computação Charles Bachman, pesquisador ligado a indústria e ganhador do prêmio ACM Alan Turing de 1973 pela sua fundamental contribuição para a área de \ac{BD} \cite{Krishna:1992}.

Devido ao alto volume de informação que é produzido e manipulado, os \acp{BD} fundamentados inicialmente por Bachman são essenciais na sociedade contemporânea. 
Segundo \citeonline{Date:1990}, uma base de dados é uma coleção de dados operacionais armazenados, usados pelos sistemas de aplicação de uma determinada organização. 
Para \citeonline{Elmasri:2011}, um \ac{BD} pode ser definido como uma abstração do mundo real, também chamado de minimundo, uma vez que representa aspectos que, em conjunto, carregam um significado implícito.

Analisando de forma mais objetiva, as bases de dados podem ser consideradas os ativos organizacionais mais importantes atualmente. 
Isso se deve ao fato de que não armazenam apenas informações triviais mas, por exemplo, também dados de faturamento, pesquisas de mercado e outros aspectos que auxiliam na tomada de decisão. 
Contudo a sua importância não está focada apenas nas organizações, sendo também possível atestar que o uso de \ac{BD} pode representar um papel crítico na vida dos usuários finais quando estes são analisados individualmente. 

Com este cenário estabelecido, é notável que existe um dever crescente da academia em fornecer um bom nível de preparo para os futuros profissionais que vão ingressar em uma indústria cada vez mais exigente. 
Com frequência as instituições de ensino superior abordam a área de \ac{BD} com disciplinas específicas e também em componentes curriculares que são convergentes nos currículos de seus cursos. 
Neste contexto, a UNIPAMPA possui cursos que envolvem desenvolvimento de software, tais como Engenharia de Software e Ciência da Computação.

%#################################################################
\section{Motivação}
%#################################################################

Segundo \citeonline{Salgado:1995}, o ensino na área de \ac{BD} é parte essencial da formação de profissionais de computação. 
O foco no ensino em \ac{BD} geralmente é dividido em quatro etapas: projeto e modelagem, sistemas de gerência de bases de dados, estudos comparativos entre estes sistemas e o desenvolvimento de aplicações. 

Tendo como premissa que existe uma crescente busca por instrumentos que apoiem o processo de ensino-aprendizagem na academia, este estudo tem foco na primeira etapa. 
O ensino de projeto e modelagem de \ac{BD} em geral é conduzido com a apresentação de tópicos essenciais e a posterior introdução ao uso de ferramentas de modelagem que utilizam abordagens geralmente gráficas. 
Este estudo tem como motivação oferecer um produto de software que dê apoio a esta fase. 
Este produto fará uso da abordagem textual, tendo em vista que possua uma gramática de fácil uso e compreensão.

%#################################################################
\section{Objetivos}
%#################################################################

O objetivo geral deste trabalho é propor a especificação e realizar a implementação de uma Linguagem Específica de Domínio, do inglês \ac{DSL}, para o projeto e modelagem de \acp{BD} relacionais. 
O foco da modelagem é em nível conceitual, mas deseja-se que a solução possa também realizar a transformação de modelos conceituais para os modelos lógico e físico. 

Para atingir o objetivo geral proposto, é fundamental que exista a divisão do problema nos seguintes objetivos específicos que precisam ser atingidos:

\begin{itemize} 
    \item Pesquisar a literatura visando encontrar propostas que façam proposições que forneçam apoio à modelagem de bases de dados; 
    \item Investigar tecnologias que auxiliem no processo de criação de linguagens específicas de domínio;
    \item Compreender quais são os requisitos necessários para a criação da linguagem;
    \item Desenvolver uma gramática para modelagem conceitual de \acp{BD} relacionais;
    \item Integrar a linguagem proposta em uma ferramenta \textit{open source};
    \item Implementar a transformação do modelo conceitual para o modelo lógico;
    \item Criar geradores para tecnologias específicas de \acp{BD}, representando assim o modelo físico;
    \item Realizar a avaliação da solução proposta;
    \item Contribuir com uma ferramenta que auxilie no processo de ensino de projeto e modelagem de \acp{BD}.
\end{itemize}

%#################################################################
\section{Questão de Pesquisa}
%#################################################################

Visando a condução do restante deste projeto de \ac{TCC} foi definida uma Questão de Pesquisa (QP) central, descrita a seguir:

\textbf{QP: Uma \ac{DSL} textual pode auxiliar, em nível conceitual de modelagem, o ensino de projeto e modelagem de banco de dados relacionais?}

%#################################################################
\section{Justificativa}
%#################################################################

Desde os primórdios os desenvolvedores utilizam texto para especificar produtos de software. 
As linguagens de programação aumentam o nível de abstração de maneira similar aos modelos. 
Logo, por consequência lógica, isso resulta em linguagens de modelagem textual.

Uma linguagem de modelagem textual é geralmente processada por mecanismos que transformam as informações expressas em formato textual para modelos. 
Esses mecanismos baseiam sua execução na estrutura sintática de uma linguagem de modelagem textual, que é formalizada em uma gramática. 
Uma gramática define palavras-chave de uma linguagem, o aninhamento de seus elementos e também a notação de suas propriedades. 
Dito isto, pode-se inferir que os modelos textuais podem trazer diversos benefícios:

\begin{itemize}
    \item \textbf{Transmitir muitos detalhes}: Quando se trata de elementos com inúmeras propriedades, a abordagem textual geralmente se destaca em relação aos gráficos. 
    O mesmo pode ser dito quanto a estruturas formadas por um grande número de partes muito pequenas, como operações matemáticas ou sequências de instruções;
    \item \textbf{Coesão}: Um modelo textual geralmente especifica os elementos inteiramente em um só local. 
    Se por um lado isso pode ser uma desvantagem para uma exibição em alto nível, em contrapartida pode facilitar a localização de definições de propriedades em baixo nível. 
    Na proposta deste trabalho toda a modelagem conceitual é realizada em apenas um arquivo;
    \item \textbf{Edição rápida}: Durante a criação e edição de modelos textuais não é necessário a recorrente alternância entre teclado e \textit{mouse}. 
    Logo, é provável que se gaste menos tempo formatando modelos textuais do que, por exemplo, refinando a posição, as ligações ou mesmo as bordas de elementos em diagramas;
    \item \textbf{Editores genéricos}: Não existe necessariamente a exigência de uma ferramenta específica para criar ou modificar modelos textuais, como é o caso de linguagens específicas de domínio com esta abordagem. 
    Para alterações simples é possível o uso de qualquer editor de texto genérico. 
    Entretanto, para tarefas maiores é melhor se ter algum suporte para a linguagem de modelagem. 
    Na proposta desse trabalho está incluso a integração da linguagem definida com um editor Eclipse, fornecendo assim um alto nível de auxílio para a escrita.
\end{itemize}


%#################################################################
\section{Organização}
%#################################################################

Este trabalho está organizado da seguinte forma. 
No \autoref{metodologiaPesquisa} é apresentado a metodologia de pesquisa adotada. 
No \autoref{fundamentacaoTeorica} é realizada a fundamentação teórica. Um mapeamento multivocal de literatura tem o protocolo e condução descritos no \autoref{mapeamentoLiteratura}. 
A seguir, ocorre a apresentação da proposta no \autoref{propostaDSL}. 
Finalmente, as considerações preliminares são discutidas no \autoref{consideracoesPreliminares}, em que ainda são apontados os trabalhos futuros e o cronograma de execução para continuação da pesquisa.