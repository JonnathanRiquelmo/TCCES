\begin{resumo}[Abstract]
%================== CONTEXTO ==================
With the advance of technology, databases have become vital elements in contemporary society.
Databases are stored data sets to describe some meaning about a specific domain.
The information stored is considered to be of great relevance in modern organizations.
In this way, the effective use of databases have great importance for the maintenance and correct progress of their activities.
%================== PROBLEMA ==================
That said, the formation in this area for professionals coming from academy must be constant, which is a fundamental point with which higher education institutions should pay special attention.
However, the variety of database systems technologies that have become available in recent years, most of which are focused on graphical approaches, make it difficult to choose entity-relationship (ER) modeling tools in industry and, consequently, in the academy.
%================== SOLUÇÃO/CONTRIBUIÇÃO ==================
In order to contribute with a relevant open source alternative, this Course Conclusion Work Project proposes a Textual Domain Specific Language (DSL) to support the teaching-learning process of conceptual database modeling.
The use of DSLs provides means to specify and model domains more quickly and productively, since they are expressive languages limited to particular domains, thus differentiating themselves from general-purpose languages.
%================== ESTADO-ARTE/PRÁTICA e MÉTODO PROPOSTO ================== 
In this sense, an investigation of the state of the art and the practice in database design and modeling using DSLs was performed.
A survey of recent innovations was carried out through a systematic mapping complemented by a survey in the gray literature.
This work covers a final set of 10 primary studies focused on DSLs and identifies 55 tools already applied in industry and academy for ER modeling at conceptual, logical and physical level.
Then, there was the selection of the Xtext framework to support the initial development of the modeling language.
%================== RESULTADOS ==================
Through this, it was possible to infer necessary requirements, design decisions and then make out the preliminary definition of a grammar.
Later, an implementation of a functional prototype and a DSL integration in an Eclipse Rich Client Platform (RCP) occurred.
In this way, there was the preliminary test of the project for this proposal. In it, the modeling process with the newly created language gained native features such as formatting, validation based on the constraints described in grammar and syntax highlighting.
The plugin could be tested due to the native integration provided by Xtext with the EMF (Eclipse Modeling Framework), a set of Eclipse features to represent models and generate equivalent code.

 \vspace{\onelineskip}
 
 \noindent 
 \textbf{Key-words}: Database. Database Design and Modeling. Conceptual Modeling. Domain Specific Language.
\end{resumo}
