\begin{resumo}[Abstract]
%================== CONTEXTO ==================
With the advance of technology, databases have become vital elements in contemporary society.
Databases are stored data sets to describe some meaning about a specific domain.
The information stored is considered to be of great relevance in modern organizations.
In this way, the effective use of databases have great importance for the maintenance and correct progress of their activities.
%================== PROBLEMA ==================
That said, the formation in this area for professionals coming from academy must be constant, which is a fundamental point with which higher education institutions should pay special attention.
However, the variety of database systems technologies that have become available in recent years, most of which are focused on graphical approaches, make it difficult to choose entity-relationship (ER) modeling tools in industry and, consequently, in the academy.
%================== SOLUÇÃO/CONTRIBUIÇÃO ==================
In order to contribute with a relevant open source alternative, this Course Conclusion Work proposes a Textual Domain Specific Language (DSL) to support the teaching-learning process of conceptual database modeling.
The use of DSLs provides means to specify and model domains more quickly and productively, since they are expressive languages limited to particular domains, thus differentiating themselves from general-purpose languages.
%================== ESTADO-ARTE/PRÁTICA e MÉTODO PROPOSTO ================== 
In this sense, an investigation of the state of the art and the practice in database design and modeling using DSLs was performed.
A survey of recent innovations was carried out through a systematic mapping complemented by a survey in the gray literature.
This work covers a final set of 10 primary studies focused on DSLs and identifies 55 tools already applied in industry and academy for ER modeling at conceptual, logical and physical level.
Then, the Xtext framework was selected to support the development of the modeling language.
After defining two versions of the grammar, a preliminary assessment was planned and carried out with a focus group of thirteen (13) participants.
With the feedback received in the dynamics, we arrived at a final version and then the plugin of the ERText solution.
To evaluate the proposal, an empirical evaluation was conducted with twenty seven (27) subjects.
The intention was to verify the effort (time), precision, recall, F-Measure, perceived utility and use of the textual approach tool compared to a graphical approach tool.
% ================== RESULTS ==================
The results show evidence that when performing modeling tasks with both approaches, there is less effort associated with the graphical approach and a very similar performance regarding the quality of the models made in both tools.

\vspace{\onelineskip}

 \noindent 
 \textbf{Key-words}: Database. Database Design and Modeling. Conceptual Modeling. Domain Specific Language.
\end{resumo}
